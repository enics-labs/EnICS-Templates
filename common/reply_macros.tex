% Extra packages:
\usepackage{clipboard} % To use \Copy \Paste commands
\usepackage{subfiles} % To call the "answers to reviewers" letter at the end
\usepackage{soul} % To highligth text
%\sethlcolor{white} %--> to get rid of the yellow hl, un-comment this

% Definition of counters for reviewers and their point-by-point concerns:
\newcounter{reviewer}
\setcounter{reviewer}{0}
\newcounter{point}[reviewer]
\setcounter{point}{0}

% Environment and Commands:
% This refines the format of how the reviewer/point reference will appear.
\renewcommand{\thepoint}{P\,\thereviewer.\arabic{point}} 

% command declarations for reviewer points and our responses
\newcommand{\reviewersection}{\stepcounter{reviewer} \bigskip \hrule
                  \section*{\large\textbf{Reviewer} \thereviewer}}

\newenvironment{point}
   {\refstepcounter{point} \bigskip \noindent {\textbf{\underline{Reviewer~Point~\thepoint}} } ---\ }
   {\par }

\newcommand{\shortpoint}[1]{\refstepcounter{point}  \bigskip \noindent 
	{\textbf{Reviewer~Point~\thepoint} } ---~#1\par }

% \newenvironment{reply}
%    {\medskip \noindent \begin{sf}\textbf{Reply}:\  }
%    {\medskip \end{sf}}
\newenvironment{reply}
   {\color{blue} \medskip \textbf{Reply}:\  }
   {\color{black} \medskip }

\newcommand{\shortreply}[2][]{\color{blue} \medskip \noindent \begin{sf}\textbf{Reply}:\  #2
	\ifthenelse{\equal{#1}{}}{}{ \hfill \footnotesize (#1)}%
	\color{black} \medskip \end{sf}}

 % \newenvironment{changes}
 %   {\medskip \noindent \begin{sf}\textbf{Authors' Action}:\  }
 %   {\medskip \end{sf}}
 \newenvironment{changes}
   {\color{blue} \medskip \textbf{Authors' Action}:\  }
   {\color{black} \medskip }   


% Commands
\newcommand{\replyFull}[2]{%

    \begin{reply}
        {\color{blue}
        #1
        }
    \end{reply}
    
    \begin{changes}
        {\color{blue}
        #2
        }
    \end{changes}
}

\newcommand{\replySingle}[1]{%

    \begin{reply}
        {\color{blue}
        #1
        }
    \end{reply}
}

%======= Counting words
%% You can pass in your own texcount params, e.g. -chinese to turn on Chinese mode, or -char to do a character count instead (which does NOT include spaces!)
%%% http://app.uio.no/ifi/texcount/documentation.html

%% To include references. DO NOT USE WITH BIBLATEX 
%TC:incbib

%% To include tabulars in main text count.
%TC:group table 0 1
%TC:group tabular 1 1

\newcommand{\detailtexcount}[1]{%
  \immediate\write18{texcount -merge -sum -q #1.tex output.bbl > #1.wcdetail }%
  \verbatiminput{#1.wcdetail}%
}

\newcommand{\quickwordcount}[1]{%
  \immediate\write18{texcount -1 -sum -merge -q #1.tex output.bbl > #1-words.sum }%
  \input{#1-words.sum} words%
}

%   -sum, -sum=   Make sum of all word and equation counts. May also use
%              -sum=#[,#] with up to 7 numbers to indicate how each of the
%              counts (text words, header words, caption words, #headers,
%              #floats, #inlined formulae, #displayed formulae) are summed.
%              The default sum (if only -sum is used) is the same as
%              -sum=1,1,1,0,0,1,1.


\newcommand{\quickcharcount}[1]{%
  \immediate\write18{texcount -1 -sum -merge -char -q #1.tex output.bbl > #1-chars.sum }%
  \input{#1-chars.sum} characters (not including spaces)%
}